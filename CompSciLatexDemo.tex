\documentclass{article} % this just tells the complilier what type of file it is
\usepackage[utf8]{inputenc} 
\usepackage{amsmath}

\title{CompSciLatexDemo}
\author{Anthony Fung}
\date{\date} % this will input the current date

\begin{document} % this is needed to begin the document

\maketitle

\section{Equations}

This is a demonstration of some of the latex commands you need for the assignment.
% if you leave a gap it will make another paragraph
For the assignment
\begin{equation}
f(x) = x_n^{2 \frac{1}{2} } \sin{(x)} \cos{(x)} \log{(x)}
\end{equation}
To write the math expression in-line (i.e. in the text), you have to put a \$ on each side of the symbol. SO if we want to represent the function f(x) in-line,  we should write it with a \$ on either side: $f(x)$
% ${insert function here}$
% the $$ is the same as [], so it will not include a number next to the equation.
% Another way of doing mathmode is using, [\ \ ] this acts as math mode
% Example: \[\mathcal{O}(N log N)\]
% this will not include a number next to the equation

% If there is a number sign next to the function you use 
% \begin{equation}          \end{equation}
% this will include a number next tot the equation


\section{Introduction}

\end{document}


% pdflatex exports it as a pdf, you just put the file 
% pdflatex {insert file name} without the {}
% pdf latex example.pdf&        - With the inclusion of "&" it will do stuff in the background